\documentclass[letter]{article}
\renewcommand{\baselinestretch}{1.25}

\usepackage[margin=1in]{geometry}
\usepackage{physics}
\usepackage{amsmath}
\usepackage{graphicx}
%\usepackage{pythonhighlight}
\usepackage{hyperref}
\usepackage{fancyvrb}

% MATLAB Formating Code
\usepackage[numbered,framed]{matlab-prettifier}
\lstset{style=Matlab-editor,columns=fullflexible}
\renewcommand{\lstlistingname}{Script}
\newcommand{\scriptname}{\lstlistingname}

% Command for easier minimization problem def
\newcommand{\optpblm}[3][eq:default]{
	\begin{equation}\label{#1}
% Array method... more centered		
%		\begin{array}{rl}
%			\text{minimize}  \hspace{0.2in} &#2 \vspace{5pt}\\
%			\text{subject to} \hspace{0.2in} &#3
%		\end{array}
% Aligned method... left aligned... idk if its better
		\begin{aligned}
			\text{minimize} \hspace{0.5in} &#2\vspace{5pt}\\
			\text{subject to \hspace{0.5in}} &#3
		\end{aligned}	
	\end{equation}
}

\newcommand{\maxpblm}[3][eq:default]{
	\begin{equation}\label{#1}
		\begin{aligned}
			\text{maximize} \hspace{0.5in} &#2\vspace{5pt}\\
			\text{subject to \hspace{0.5in}} &#3
		\end{aligned}	
	\end{equation}
}
\allowdisplaybreaks

\title{MECH 6327 - Homework 3}
\author{Jonas Wagner}
\date{2021, March 24}




\begin{document}

\maketitle

\newpage
\tableofcontents

\newpage
\section*{BV Textobook Problems}
\subsection{Problem 5.43}
The dual a SOCP defined as:
\optpblm{f^T x}{\norm{A_i x + b_i}_2 \leq c_i^T x + d_i, \ i = 1,\dots,m}
with $x \in \real^n$ can be expressed as:
\maxpblm{\sum_{i=1}^m \qty(b_i^T u_i - d_i v_i)}{
	  \sum_{i=1}^m \qty(A_i^T u_i - c_i v_i) + f = 0\\
	& \norm{u_i}_2 \leq v_i, \ i = 1,\dots,m}
with variables $u_i \in \real^n_i$, $v_i \in \real, \ i=1,\dots,m$ and problem data $f\in \real^n, A_i \in \real^{n_i \cross n}, b_i \in \real^{n_i}, c_i \in \real, \ i = 1,\dots,m$.\\

\subsubsection{Part a}
\textbf{Problem:}
Derive the dual by defining $y_i \in \real^{n_i}$ and $t_i \in \real$ and the inequalities $y_i = A_i x + b_i, \ t_i = c_i^T x + d_i$ then deriving the Lagrange dual.\\

\noindent
\textbf{Solution:}








\subsubsection{Part b}
\textbf{Problem:}
Start with the conic formulation of the SOCP and use the conic dual to prove the equivelence. Use the fact that the secound-order dual is self-dual.\\

\noindent
\textbf{Solution:}
















\newpage
\section{Problem 1: Robust control design}
For the standard DT dynamical system defined as:
\begin{equation}
	x_{t+1} = A x_t + B u_t
\end{equation}
with dynamic matrix $A$ unknown but assumed to belong to a set:
\begin{equation}
	A \in \mathcal{A} = \text{conv}(A_1,\dots,A_m)
\end{equation}
with $A_i$ and $B$ known.\\


\textbf{Problem:}
For a state-feedback controller $u_t = K x_t$ use Lyapunov techniques to design it so the system is globally asymptotically stable (GAS) by solving a semi-definite program (SDP).\\

\textbf{Solution:}










\newpage
\section{Problem 2: Nonnegative and sum of squares polynomials}
The Motzkin polynomial is defined as:
\begin{equation}
	M(x,y) = x^2 y^4 + x^4 y^2 + 1 - 3 x^2 y^2
\end{equation}

\textbf{Problem:}
Show that the Motzkin polynomial is nonnegative but can be expressed as sum of squares. It is sufficient to show this using numerical and/or symbolic solvers.\\

\textbf{Solution:}











%\newpage
%\appendix
%\section{MATLAB Code:}\label{apx:matlab}
%All code I write in this course can be found on my GitHub repository:\\
%\href{https://github.com/jonaswagner2826/MECH6337}{https://github.com/jonaswagner2826/MECH6327}
%\lstinputlisting[caption={MECH6327\_HW3},label={script:HW3}]{MECH6327_HW4.m}


\end{document}