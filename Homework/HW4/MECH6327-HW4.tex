\documentclass[letter]{article}
\renewcommand{\baselinestretch}{1.25}

\usepackage[margin=1in]{geometry}
\usepackage{physics}
\usepackage{amsmath}
\usepackage{graphicx}
%\usepackage{pythonhighlight}
\usepackage{hyperref}
\usepackage{fancyvrb}

% MATLAB Formating Code
\usepackage[numbered,framed]{matlab-prettifier}
\lstset{style=Matlab-editor,columns=fullflexible}
\renewcommand{\lstlistingname}{Script}
\newcommand{\scriptname}{\lstlistingname}

% Command for easier minimization problem def
\newcommand{\optpblm}[3][eq:default]{
	\begin{equation}\label{#1}
% Array method... more centered		
%		\begin{array}{rl}
%			\text{minimize}  \hspace{0.2in} &#2 \vspace{5pt}\\
%			\text{subject to} \hspace{0.2in} &#3
%		\end{array}
% Aligned method... left aligned... idk if its better
		\begin{aligned}
			\text{minimize} \hspace{0.5in} &#2\vspace{5pt}\\
			\text{subject to \hspace{0.5in}} &#3
		\end{aligned}	
	\end{equation}
}

\newcommand{\maxpblm}[3][eq:default]{
	\begin{equation}\label{#1}
		\begin{aligned}
			\text{maximize} \hspace{0.5in} &#2\vspace{5pt}\\
			\text{subject to \hspace{0.5in}} &#3
		\end{aligned}	
	\end{equation}
}
\allowdisplaybreaks

\title{MECH 6327 - Homework 4}
\author{Jonas Wagner}
\date{2021, April 12}




\begin{document}

\maketitle

\newpage
\tableofcontents

\newpage
\section*{BV Textobook Problems}
\subsection{Problem 5.43}
The dual a SOCP defined as:
\optpblm{f^T x}{\norm{A_i x + b_i}_2 \leq c_i^T x + d_i, \ i = 1,\dots,m}
with $x \in \real^n$ can be expressed as:
\maxpblm{\sum_{i=1}^m \qty(b_i^T u_i - d_i v_i)}{
	  \sum_{i=1}^m \qty(A_i^T u_i - c_i v_i) + f = 0\\
	& \norm{u_i}_2 \leq v_i, \ i = 1,\dots,m}
with variables $u_i \in \real^n_i$, $v_i \in \real, \ i=1,\dots,m$ and problem data $f\in \real^n, A_i \in \real^{n_i \cross n}, b_i \in \real^{n_i}, c_i \in \real, \ i = 1,\dots,m$.\\

\newpage
\subsubsection{Part a}
\textbf{Problem:}
Derive the dual by defining $y_i \in \real^{n_i}$ and $t_i \in \real$ and the equalities $y_i = A_i x + b_i, \ t_i = c_i^T x + d_i$ then deriving the Lagrange dual.\\

\noindent
\textbf{Solution:}
The problem can first be written in a standard form as:
\optpblm{f^T}{
	y_i = A_i x + b_i\\
	&t_i = c_i^T x + d_i\\
	&\norm{y_i}_2 \leq t_i, \ \forall i = 1, \dots, m}

The lagrange can then be defined with
\begin{align}
	L(x,y,t, \lambda_i, \nu_i, \mu_i)
	&= f^T 
	+ \sum_{i=1}^m \lambda_i \qty(\norm{y_i}_2 - t_i)
	+ \sum_{i=1}^m \mu_i^T (t_i - C_i x + d_i)\\
	&= \qty(f + \sum_{i=1}^m \qty(A_i^T \mu_i - c_i \nu_i))^T x
	+ \sum_{i=1}^m \lambda_i \norm{y_i}_2 + \mu_i^T y_i \nonumber\\
	&\ \ \ + \sum_{i=1}^m \qty(-\lambda_i + \nu_i) t_i
	- \qty(\sum_{i=1}^m b_i^T \mu_i - d_i \nu_i)
\end{align}

Since the definition of the dual optimization problem is to maximize $$g(\lambda_i, \nu_i, \mu_i) = \inf_{x,y_i,t_i} L(x,y,t, \lambda_i, \nu_i, \mu_i)$$ the $\inf$ can be found by determining when a min/max would occur for each of the variables.

For the critical point on $x$ the direvative can be set to zero and thus the following equality must hold:
\begin{equation}
	f + \sum_{i=1}^m \qty(A_i^T \mu_i - c_i \nu_i) = 0
\end{equation}

For the $y_i$ related term, it is known $$\sum_{i=1}^m \lambda_i \norm{y_i}_2 + \mu_i^T y_i$$ will be bounded below if it is within the cone defined by $\lambda_i \norm{y_i}_2 \geq \norm{\mu_i}_2 y_i$ which can be rewritten as: $$\norm{\mu_i}_2 \leq \lambda_i$$

For the critical point over $t_i$ the equality $\nu_i = \lambda_i$.

From this the dual problem can be obtained when the quantity $\qty(\sum_{i=1}^m b_i^T \mu_i - d_i \nu_i)$ is maximized.

Thus the dual problem is defined as:
\maxpblm{\sum_{i=1}^m \qty(b_i^T u_i - d_i v_i)}{
	\sum_{i=1}^m \qty(A_i^T u_i - c_i v_i) + f = 0\\
	& \norm{u_i}_2 \leq v_i, \ i = 1,\dots,m}

\newpage
\subsubsection{Part b}
\textbf{Problem:}
Start with the conic formulation of the SOCP and use the conic dual to prove the equivalence. Use the fact that the second-order dual is self-dual.\\

\noindent
\textbf{Solution:}
Starting with the SOCP given as
\optpblm{f^T x}{\norm{A_i x + b_i}_2 \leq c_i^T x + d_i, \ i = 1,\dots,m}

a standard form can be defined by
\optpblm{f^T x}{\qty(A_i x + b_i, c_i^T x + d_i) \preceq_{2} 0}


Since the conic dual transformation is known to transform
\optpblm{f^T x}{- \qty(A^T x + b, c^T x + d) \preceq_{K} 0}

into its dual according to its dual cone definition
\maxpblm{b^T u + d \ v}{A^T u + v c = f\\ &(u,v) \succeq_{K^*} 0, \ i = 1,\dots,m}

and from the fact that the 2-norm is self-dual, the dual program is given as
\maxpblm{\sum_{i=1}^m \qty(b_i^T u_i - d_i v_i)}{
	- \sum_{i=1}^m \qty(A_i^T u_i - c_i v_i) = f\\
	& (u_i, v_i) \succeq_{2} 0, \ i = 1,\dots,m}

or equivalently
\maxpblm{\sum_{i=1}^m \qty(b_i^T u_i - d_i v_i)}{
	\sum_{i=1}^m \qty(A_i^T u_i - c_i v_i) + f = 0\\
	& \norm{u_i}_2 \leq v_i, \ i = 1,\dots,m}


\newpage
\section{Problem 1: Robust control design}
For the standard DT dynamical system defined as:
\begin{equation}
	x_{t+1} = A x_t + B u_t
\end{equation}
with dynamic matrix $A$ unknown but assumed to belong to a set:
\begin{equation}
	A \in \mathcal{A} = \text{conv}(A_1,\dots,A_m)
\end{equation}
with $A_i$ and $B$ known.\\


\textbf{Problem:}
For a state-feedback controller $u_t = K x_t$ use Lyapunov techniques to design it so the system is globally asymptotically stable (GAS) by solving a semi-definite program (SDP).\\

\textbf{Solution:}
The closed-loop system for the DT dynamical system can be defined by the dynamics
\begin{equation}
	x_{t+1} = \hat{A} x = (A + B K) x
\end{equation}
where $\hat{A} = A + B K$.

In order for the closed-loop system to be Globally Asymptotically Stable, a quadratic Lyapnov Function could be used to prove that if the following inequality is true then the system is GAS:
\begin{equation}
	\hat{A}^T P \hat{A} - P \prec 0
\end{equation}

Since the system dynamics themselves are uncertain, this inequality will not be enough to prove GAS. This can be be address, however, by considering all $A \in \mathcal{A}$ to be a linear combination of the individual corner matrices. Since this is a convex hull, it is known that following this to its conclusion, GAS can be guaranteed for all $A \in \mathcal{A}$ if $A_i$ is GAS $\forall i = 1, \dots, m$.

Following this, a stabalizing gain can then be found as follows:
\begin{align}
	\hat{A}_i^T P \hat{A}_i - P \prec 0
\end{align}
recognizing the Schur's compliment form, the following is true
\begin{align}
	\mqty[P & \hat{A}_i^T\\ \hat{A}_i P^{-1}] \succ 0\\
	\mqty[P & (A_i + B K)^T\\ (A_i + B K) &P^{-1}] \succ 0
\end{align}

A SDP feasibility problem can then  be done to solve the problem such that
\begin{equation}
	\begin{aligned}
		\mqty[P & \hat{A}_i^T\\ \hat{A}_i P^{-1}] \succ 0\\
		B^{-1} \qty(\hat{A}_i - A_i) - K = 0, \ \forall i = 1, \dots, m
	\end{aligned}
\end{equation}
with variables $P$, $\hat{A}_i$, and $K$, along with problem data $A_i$ and $B$.
This is a problem that can now be easily implemented using CVX or YALMIP in MATLAB for given problem data.


\newpage
\section{Problem 2: Nonnegative and sum of squares polynomials}
The Motzkin polynomial is defined as:
\begin{equation}
	M(x,y) = x^2 y^4 + x^4 y^2 + 1 - 3 x^2 y^2
\end{equation}

\textbf{Problem:}
Show that the Motzkin polynomial is nonnegative but can be expressed as sum of squares. It is sufficient to show this using numerical and/or symbolic solvers.\\

\textbf{Solution:}
Nonegativity of the Motzkin polynomial can be proven using the AM-GM inequality using $n=3$ with $x^4y^2, x^2 y^4, 1$.
\begin{align}
	\sqrt[n]{\prod_{i=1}^{n} x_i} &\leq \frac{1}{n} \sum_{i=1}^n x_i\\
	\sqrt[3]{(x^4 y^2)(x^2 y^4)(1)} &\leq \frac{1}{3} (x^4 y^2 + x^2 y^4 + 1)\\
	x^2 y^2 &\leq \frac{1}{3} (x^4 y^2 + x^2 y^4 + 1)\\
	3 x^2 y^2 &\leq x^4 y^2 + x^2 y^4 + 1\\
	x^4 y^2 + x^2 y^4 + 1 - 3 x^2 y^2 &\geq 0
\end{align}
Therefore the Motzkin polynomial is non-negative.

However, it is not possible to put this into sum of square form using the standard solver as demonstrated by the infeasability result from the solvesos() command in yalmip (shown in \appendixname \ref{apx:matlab})










\newpage
\appendix
\section{MATLAB Code:}\label{apx:matlab}
All code I write in this course can be found on my GitHub repository:\\
\href{https://github.com/jonaswagner2826/MECH6337}{https://github.com/jonaswagner2826/MECH6327}
\lstinputlisting[caption={MECH6327\_HW4},label={script:HW4}]{MECH6327_HW4.m}



\newpage
\textbf{Refrences:} * not bibtex becouse of time...

https://people.eecs.berkeley.edu/~elghaoui/Teaching/EE227A/lecture10.pdf

https://people.orie.cornell.edu/miketodd/iccopt.pdf



\end{document}