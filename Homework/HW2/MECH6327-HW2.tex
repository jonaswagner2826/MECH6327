\documentclass[letter]{article}
\renewcommand{\baselinestretch}{1.25}

\usepackage[margin=1in]{geometry}
\usepackage{physics}
\usepackage{amsmath}
\usepackage{graphicx}
\usepackage{pythonhighlight}

\allowdisplaybreaks

%opening
\title{MECH 6327 - Homework 2}
\author{Jonas Wagner}
\date{2021, February 22}

\begin{document}

\maketitle


\section{Problem Set 1: Convex Sets}

\subsection{Problem 2.5}
\textbf{Problem:}\\
What is the distance between two parallel hyperplanes: $\{x \in \real^n | a^T x = b_1\}$ and $\{x \in \real^n | a^T x = b_2\}$?\\

\noindent
\textbf{Solution:}\\
Under the assumption that $a\in \real^n$ and $b_1,b_2 \in \real$, the quantity $a^T x_0$ represents the component of $x_0$ in the normal direction. Similarly, the quantities $b_1$ and $b_2$ represent the euclidean distance of the hyperplane from the origin (in the normal direction). Since the hyperplanes are parrellel, the distance between them is the difference between their offsets:
\begin{equation}
	\text{Distance between hyperplanes: } b_1 - b_2
\end{equation}


\subsection{Problem 2.7}
\textbf{Problem:}\\
\textit{Voronoi description of halfspace.} Let $a$ and $b$ be distinct points in $\real^n$. Show that the set of all points that are closer to $a$ than $b$ via the euclidean norm is a halfspace. Describe it explicitly as an inequality and draw a picture.

\noindent
\textbf{Solution:}\\
The set of all points closer to $a$ then $b$ can be defined as:
\begin{equation}
	\{x \in \real^n \ | \ \norm{x-a}_2 \leq \norm{x-b}_2\}
\end{equation}

The boundary defining this halfspace will be a plane defined by the normal vector $c$ representing the distance between $a$ and $b$, and the offset coefficient $d$ describing intersection of the plane through the half-way point between $a$ and $b$. The quantities $c$ and $d$ can therefore be defined by:
\begin{equation}
	\begin{aligned}
		c &= b - a\\
		d &= \frac{c^T a + c^T b}{2}\\
		  &= \frac{1}{2} c^T (a+b)
	\end{aligned}
\end{equation}

The halfspace, that is equivenlent to $x$, can be described by the following:
\begin{equation}
	\{x\in \real^n \ | \ c^T x \leq d\}
\end{equation}

This can be visualized in two dimensions 


\subsection{Problem 2.12}
\textbf{Problem:}\\
Multiple choice problem from book....


\textbf{Solution:}\\



\subsection{Problem 2.28}
\textbf{Problem:}\\



\textbf{Solution:}\\




\subsection{Problem 2.33}
\textbf{Problem:}\\



\textbf{Solution:}\\





\section{Problem Set 2: Convex Functions}

\subsection{Problem 3.6}
\textbf{Problem:}\\



\textbf{Solution:}\\


\subsection{Problem 3.16}
\textbf{Problem:}\\



\textbf{Solution:}\\


\subsection{Problem 3.18a}
\textbf{Problem:}\\



\textbf{Solution:}\\


\subsection{Problem 3.22}
\textbf{Problem:}\\



\textbf{Solution:}\\

\end{document}
